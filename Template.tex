% Copyright (c) 2019 Bochen Tan
% Public domain.
%本模板的宗旨是尽量绿色,不需要附加安装任何东西。
%按照教务部下发的WORD说明文档格式,下简称“说明”
%没有封面和评阅表,这两部分请直接在Cover&ReviewTable.doc中写再输出pdf拼到一起
%doc小改动:封面校徽和文字替换为了高清版本,“题目:”和中文题目对齐,中英文题目分在了表的两行
%doc小改动:插入了两个白页,使得连续打印的时候封面和表格都在奇数页
%正文部分改动:在每一页下方中央加了页码,因为说明中页眉不分奇偶页,所以页码就都在中央吧
%不含自动的参考文献,因为说明中参考文献格式不典型,请手动输入或自行写程序
%在Windows或Linux下渲染出字体更接近说明,Mac OS上字体不太一样
%有警告\headheight is too small,fancyhdr的上距离有点小,似乎问题不大

\documentclass[UTF8,openany,AutoFakeBold,AutoFakeSlant,cs4size]{ctexbook}
%openany 使一章可以从偶数页开始,因为说明中每一章并没有只能从奇数页开始,虽然这是常理
%AutoFakeBold 和 AutoFakeSlant 因为 CJK 里没有真正的加粗和倾斜,如果额外字体则效果更好
%cs4size 因为要求主题是小四号字

\usepackage[a4paper,left=3.18cm,right=3.18cm,top=2.54cm,bottom=2.54cm]{geometry}
%office中正常页边距



\usepackage{amsmath}
\usepackage{bm}
\usepackage{amsfonts}
\usepackage{enumerate}
\usepackage{fancyhdr}



\usepackage{cite}
\newcommand{\upcite}[1]{\textsuperscript{\cite{#1}}} %引用在右上角



\usepackage{multirow,booktabs,makecell}
\usepackage{graphicx}
\usepackage[font=small,labelsep=space]{caption} %五号,宋体/Time new roman
\renewcommand{\thetable}{\arabic{table}} %表格和图片编号不分章节,直接1,2,3 ...
\renewcommand{\thefigure}{\arabic{table}}



\usepackage{tocloft} %自定义目录,说明中没有明确规定,和WORD自动生成目录格式一致

%“全文目录”四个字的格式
\renewcommand\cftbeforetoctitleskip{0pt}
\renewcommand\cftaftertoctitleskip{0pt}
\renewcommand\cfttoctitlefont{\bfseries\heiti\zihao{2}}

\renewcommand\cftchapfont{\heiti\normalsize} %黑体小四
\renewcommand\cftchapdotsep{\cftdotsep} %有点连到页码,点间距不确定,待改
\renewcommand\cftchappagefont{\songti\normalsize} %宋体小四页码
\renewcommand\cftbeforechapskip{0pt}

%1. 第一级 五号宋体,缩进两个字符,页码一致
\renewcommand\cftsecfont{\songti\small}
\renewcommand\cftsecpagefont{\songti\small}
\renewcommand\cftsecaftersnum{.} %一级目录号后加点
\renewcommand\cftsecindent{2em}
\renewcommand\cftbeforesecskip{0pt}

%1.1 第二级 五号宋体,缩进四个字符,页码一致
\renewcommand\cftsubsecfont{\songti\small}
\renewcommand\cftsubsecpagefont{\songti\small}
\renewcommand\cftsubsecindent{4em}
\renewcommand\cftbeforesubsecskip{0pt}

%1.1.1 第二级 五号宋体,缩进四个字符,页码一致
\renewcommand\cftsubsubsecfont{\songti\small}
\renewcommand\cftsubsubsecpagefont{\songti\small}
\renewcommand\cftsubsubsecindent{4em}
\renewcommand\cftbeforesubsubsecskip{0pt}



\usepackage{titlesec}%自定义章节标题
\CTEXsetup[format={\bfseries\center\heiti\zihao{2}},beforeskip=0pt]{chapter}
%第一章  绪论(二号、黑体) beforeskip为上方垂直距离看起来还比说明偏大,待改

\setcounter{tocdepth}{3}
\setcounter{secnumdepth}{3}
%使目录中有三级标题,即subsubsection

\renewcommand\thesection{\arabic{section}} % 使得不显示章名,只显示节名
\titleformat{\section}
{\raggedright\zihao{3}\bfseries\songti}
{\thesection.\quad}
{0pt}
{}%1. 第一级(三号、宋体/Time new roman、加粗)

\titleformat{\subsection}
{\raggedright\bfseries\zihao{4}\songti}
{\thesubsection\quad}
{0pt}
{}%1.1 第二级(四号,宋体/Time new roman,加粗)

\titleformat{\subsubsection}
{\raggedright\bfseries\zihao{-4}\songti}
{\thesubsubsection\quad}
{0pt}
{}%1.1.1 第三级(小四,宋体/Time new roman,加粗)




% 封面依赖的宏包
\usepackage{xcoffins} % 用于设计封面格式
\usepackage{xcolor}
\usepackage{xeCJK} % 用于引入楷体
\usepackage{soul} % 用于设置下划线宽度
\setul{}{2pt}
\setmainfont{Times New Roman} % Times New Roman 作为默认英文字体
% 引入楷体,请改成自己系统里对应的名字
\setCJKfamilyfont{kaiti}[AutoFakeBold=1.5]{楷体}
\newcommand{\kaiti}{\CJKfamily{kaiti}}

% 评阅表依赖的宏包
\input{ReviewTableHead}



\title{}
\author{}
\date{}
\begin{document}

% 封面中需要修改的内容直接在此处更改即可
\newcommand{\chineseTitle}{中文题目(楷体,二号,加粗)}
\newcommand{\englishTitle}{英文题目(Times New Roman,三号,加粗)}
\newcommand{\name}{张三}
\newcommand{\studentID}{15000xxxxx}
\newcommand{\school}{信息科学技术学院}
\newcommand{\major}{电子信息科学与技术}
\newcommand{\advisor}{李四}
% 插入封面
% 声明需要的Coffin
\NewCoffin \result
\NewCoffin \topBox
\NewCoffin \badge
\NewCoffin \pku
\NewCoffin \headingText
\NewCoffin \titleText
\NewCoffin \chineseTitleText
\NewCoffin \englishTitleText
\NewCoffin \nameText
\NewCoffin \studentIDText
\NewCoffin \schoolText
\NewCoffin \majorText
\NewCoffin \advisorText
\NewCoffin \dateText


% 各个Coffin的内容
\SetHorizontalCoffin \result {}
\SetHorizontalCoffin \topBox {\color{white} \rule{210mm}{41mm}}
\SetHorizontalCoffin \badge {\includegraphics[width=22mm]{校徽.png}}
\SetHorizontalCoffin \pku {\includegraphics[width=62mm]{中文校名_红色.png}}
\SetVerticalCoffin \headingText{160mm}{\center\heiti\fontsize{36}{36}\textcolor{black}{本科生毕业论文}}
\SetVerticalCoffin \titleText{25.4mm}{\bfseries\songti\fontsize{16pt}{16pt}{题目:}}
\fontsize{22pt}{22pt}\selectfont
\SetVerticalCoffin \chineseTitleText{110mm}{\bfseries\kaiti\underline{\makebox[112mm][l]{\chineseTitle}}}
\fontsize{16pt}{16pt}\selectfont
\SetVerticalCoffin \englishTitleText{110mm}{\bfseries\underline{\makebox[112mm][l]{\englishTitle}}}
\SetVerticalCoffin \nameText{104mm}{\center\songti\fontsize{16}{16}\textcolor{black}{姓\qquad 名:\underline{\makebox[76mm][c]{\kaiti{\name}}}}}
\SetVerticalCoffin \studentIDText{104mm}{\center\songti\fontsize{16}{16}\textcolor{black}{学\qquad 号:\underline{\makebox[76mm][c]{\kaiti\fontsize{16pt}{16pt}{\studentID}}}}}
\SetVerticalCoffin \schoolText{104mm}{\center\songti\fontsize{16}{16}\textcolor{black}{院\qquad 系:\underline{\makebox[76mm][c]{\kaiti{\school}}}}}
\SetVerticalCoffin \majorText{104mm}{\center\songti\fontsize{16}{16}\textcolor{black}{本科专业:\underline{\makebox[76mm][c]{\kaiti{\major}}}}}
\SetVerticalCoffin \advisorText{104mm}{\center\songti\fontsize{16}{16}\textcolor{black}{指导老师:\underline{\makebox[76mm][c]{\kaiti{\advisor}}}}}
\fontsize{18}{18}\selectfont
\SetVerticalCoffin \dateText{104mm}{\center\kaiti\textcolor{black}{二〇一玖\quad 年\quad 五\quad 月}}


% 指定各个Coffin相对位置关系
\JoinCoffins \result \topBox
\JoinCoffins \result[\topBox-hc, \topBox-b] \badge[r, b](-28mm, -20.6mm)
\JoinCoffins \result[\topBox-hc, \topBox-b] \pku[l, b](-11.3mm, -18.6mm)
\JoinCoffins \result[\topBox-hc, \topBox-b] \headingText[hc, b](0mm, -63.8mm)
\JoinCoffins \result[\headingText-hc, \headingText-b] \titleText[l, t](-73.25mm, -20mm)
\JoinCoffins \result[\headingText-hc, \headingText-b] \chineseTitleText[l, t](-49.85mm, -18.85mm)
\JoinCoffins \result[\headingText-hc, \headingText-b] \englishTitleText[l, t](-49.85mm, -34mm)
\JoinCoffins \result[\headingText-hc, \headingText-b] \nameText[hc, t](0mm, -60mm)
\JoinCoffins \result[\nameText-hc, \nameText-b] \studentIDText[hc, t](0mm, 0mm)
\JoinCoffins \result[\studentIDText-hc, \studentIDText-b] \schoolText[hc, t](0mm, 0mm)
\JoinCoffins \result[\schoolText-hc, \schoolText-b] \majorText[hc, t](0mm, 0mm)
\JoinCoffins \result[\majorText-hc, \majorText-b] \advisorText[hc, t](0mm, 0mm)
\JoinCoffins \result[\advisorText-hc, \advisorText-b] \dateText[hc, t](0mm, -20mm)


% 输出封面
\thispagestyle{empty}
\newgeometry{left=0mm,bottom=0mm, top=0mm, right=0mm}
\noindent\TypesetCoffin \result
\restoregeometry
\clearpage

% 插入导师评阅表
\thispagestyle{empty}
\newgeometry{left=2cm, right=2cm, top=2.64cm, bottom=2.54cm}
\renewcommand\arraystretch{1.2}

\begin{center}
{\songti\fontsize{16pt}{16pt}{北京大学本科毕业论文导师评阅表}}
\end{center}

\begin{table}[H]
	\centering
    \begin{tabular}{|rrrrrr|}
    \hline
    \multicolumn{1}{|p{4em}|}{学生姓名} & \multicolumn{1}{p{3em}|}{} & \multicolumn{1}{p{5em}|}{学生学号} & \multicolumn{1}{p{6.5em}|}{} & \multicolumn{1}{p{6.565em}|}{论文成绩} &  \multicolumn{1}{r|}{}\\
    \hline
    \multicolumn{1}{|p{4em}|}{学院(系)} & \multicolumn{3}{r|}{} & \multicolumn{1}{p{6.565em}|}{学生所在专业} &  \\
    \hline
    \multicolumn{1}{|r|}{\multirow{2}[2]{*}{导师姓名}} & \multicolumn{1}{r|}{\multirow{2}[2]{*}{}} & \multicolumn{1}{p{5em}|}{导师单位/} & \multicolumn{1}{r|}{\multirow{2}[2]{*}{}} & \multicolumn{1}{p{6.565em}|}{\multirow{2}[2]{*}{导师职称}} & \multirow{2}[2]{*}{} \\
    \multicolumn{1}{|r|}{} & \multicolumn{1}{r|}{} & \multicolumn{1}{p{5em}|}{所在研究所} & \multicolumn{1}{r|}{} & \multicolumn{1}{r|}{} &  \\
    \hline
    \multicolumn{2}{|p{9em}|}{\centering{论文题目}} & \multicolumn{4}{r|}{\multirow{2}[2]{*}{}} \\
    \multicolumn{2}{|p{9em}|}{\centering{(中、英文)}} & \multicolumn{4}{r|}{} \\
    \hline
    \multicolumn{6}{|p{35.88em}|}{\center{导师评语}} \\
    \multicolumn{6}{|p{35.88em}|}{\kaiti{(包含对论文的性质、难度、分量、综合训练等是否符合培养目标的目的等评价)}} \\
    \multicolumn{6}{|c|}{} \\
    \multicolumn{6}{|c|}{} \\
    \multicolumn{6}{|c|}{} \\
    \multicolumn{6}{|c|}{} \\
    \multicolumn{6}{|c|}{} \\
    \multicolumn{6}{|c|}{} \\
    \multicolumn{6}{|c|}{} \\
    \multicolumn{6}{|c|}{} \\
    \multicolumn{6}{|r|}{} \\
    \multicolumn{6}{|r|}{} \\
    \multicolumn{6}{|r|}{} \\
    \multicolumn{6}{|r|}{} \\
    \multicolumn{6}{|r|}{} \\
    \multicolumn{6}{|r|}{} \\
    \multicolumn{6}{|r|}{} \\
    \multicolumn{6}{|p{35.88em}|}{                                                                             \hfill 导师签名:\qquad\qquad\qquad\qquad\qquad\qquad\qquad\qquad } \\
    \multicolumn{6}{|r|}{} \\
    \multicolumn{6}{|p{35.88em}|}{\hfill 年 \qquad\quad 月 \qquad\quad 日 \qquad\qquad\qquad} \\
    \multicolumn{6}{|r|}{} \\
    \hline
    \end{tabular}
\end{table}

\renewcommand\arraystretch{1}
\restoregeometry
\clearpage


\linespread{1.5}\selectfont
\chapter*{版权声明}
\setcounter{page}{0}
% 本页不计页码
\thispagestyle{empty}
% 本页无页眉和页脚
任何收存和保管本论文各种版本的单位和个人,未经本论文作者同意,不得将本论文转借他人,亦不得随意复制、抄录、拍照或以任何方式传播。否则,引起有碍作者著作权之问题,将可能承担法律责任。
\clearpage

%版权声明后空白一页,使得摘要从奇数页开始。
\quad
\setcounter{page}{0}
% 本页不计页码
\thispagestyle{empty}
% 本页无页眉和页脚
\clearpage



\pagestyle{fancy}
\normalsize
\linespread{1.5}\selectfont
%小四号,宋体/Time new roman,1.5倍行距
\chapter*{摘要}
Copyright (c) 2019 Bochen Tan

Public domain.

本模板的宗旨是尽量绿色,不需要附加安装任何东西。

按照教务部下发的WORD说明文档格式,下简称“说明”

没有封面和评阅表,这两部分请直接在Cover\&ReviewTable.doc中写再输出pdf拼到一起

doc小改动:封面校徽和文字替换为了高清版本,“题目:”和中文题目对齐,中英文题目分在了表的两行

doc小改动:插入了两个白页,使得连续打印的时候封面和表格都在奇数页

正文部分改动:在每一页下方中央加了页码,因为说明中页眉不分奇偶页,所以页码就都在中央吧

不含自动的参考文献,说明中参考文献格式不典型,请手动输入或自行写程序

在Windows或Linux下渲染出字体更接近说明,Mac OS上字体不太一样

有警告$\backslash$ headheight is too small,fancyhdr的上距离有点小,似乎问题不大

\bigskip
\noindent{\bfseries\songti 关键词: } 



\addcontentsline{toc}{chapter}{摘要} %手动加入目录
\fancypagestyle{plain} %因为latex默认每章第一页是plain所以需要重置一下plain和说明统一
{
	\fancyhf{} %清空
	
	\fancyhead[RE,RO]{摘要}
	%偶数页右页眉,奇数页右页眉均为“摘要”,及章名\leftmark

	\fancyhead[LE,LO]{北京大学本科生毕业论文}
	%偶数页左页眉,奇数页左页眉均为“北京大学本科生毕业论文”
	
	\fancyfoot[CO,CE]{~\thepage~}
	%偶数页和奇数页中页脚为页码,从对称考虑,因为每页在说明中都是一样的,不分奇偶
	
	\renewcommand{\headrulewidth}{0.7pt} %页眉线宽度,可调,不太清楚说明中是多少,待改
	
	\renewcommand{\footrulewidth}{0pt} %页脚线宽度为0,既没有
}

%默认的风格是fancy,设置于下,用于每章非第一页
\fancyhf{}
\fancyhead[RE,RO]{摘要}
\fancyhead[LE,LO]{北京大学本科生毕业论文}
\fancyfoot[CO,CE]{~\thepage~}
\renewcommand{\headrulewidth}{0.7pt}
\renewcommand{\footrulewidth}{0pt}
\clearpage






\small
\linespread{1.5}\selectfont
%5号,Time new roman,1.5倍行距
\chapter*{\bfseries Abstract}

\bigskip
\noindent
{\bfseries Key Words: }


	
\addcontentsline{toc}{chapter}{\bfseries Abstract} %Abstract加粗
\fancypagestyle{plain}
{
	\fancyhf{}
	\fancyhead[RE,RO]{Abstract}
	\fancyhead[LE,LO]{北京大学本科生毕业论文}
	\fancyfoot[CO,CE]{~\thepage~}
	\renewcommand{\headrulewidth}{0.7pt}
	\renewcommand{\footrulewidth}{0pt}
}
\fancyhf{}
\fancyhead[RE,RO]{Abstract}
\fancyhead[LE,LO]{北京大学本科生毕业论文}
\fancyfoot[CO,CE]{~\thepage~}
\renewcommand{\headrulewidth}{0.7pt}
\renewcommand{\footrulewidth}{0pt}
\clearpage





\fancypagestyle{plain}
{
	\fancyhf{}
	\fancyhead[RE,RO]{全文目录}
	\fancyhead[LE,LO]{北京大学本科生毕业论文}
	\fancyfoot[CO,CE]{~\thepage~}
	\renewcommand{\headrulewidth}{0.7pt}
	\renewcommand{\footrulewidth}{0pt}
}
\fancyhf{}
\fancyhead[RE,RO]{全文目录}
\fancyhead[LE,LO]{北京大学本科生毕业论文}
\fancyfoot[CO,CE]{~\thepage~}
\renewcommand{\headrulewidth}{0.7pt}
\renewcommand{\footrulewidth}{0pt}
\renewcommand{\contentsname}{\centerline{全文目录}}
\tableofcontents
\addcontentsline{toc}{chapter}{全文目录}
\clearpage





\normalsize
\linespread{1.5}\selectfont
%正文,小四号,中文宋体,英文Time new roman,1.5倍行距
\fancypagestyle{plain}
{
	\fancyhf{}
	\fancyhead[RE,RO]{\leftmark}
	\fancyhead[LE,LO]{北京大学本科生毕业论文}
	\fancyfoot[CO,CE]{~\thepage~}
	\renewcommand{\headrulewidth}{0.7pt}
	\renewcommand{\footrulewidth}{0pt}
}
\fancyhf{}
\fancyhead[RE,RO]{\leftmark}
\fancyhead[LE,LO]{北京大学本科生毕业论文}
\fancyfoot[CO,CE]{~\thepage~}
\renewcommand{\headrulewidth}{0.7pt}
\renewcommand{\footrulewidth}{0pt}



\chapter{章节名称}
\section{一级段落名称}
\subsection{二级段落名称}
\subsubsection{三级段落名称}
引用如\upcite{PhysRev.47.777},引用表如表\ref{tab:input_output_r},引用图如图\ref{fig:sample}.

\begin{table}[h]
\small %内容,(五号,宋体/Time new roman)
\centering
\caption{不同频率下的输入和输出阻抗}
\label{tab:input_output_r}
\begin{tabular}{cccc} %表格使用三线表
\toprule %不确定说明中三条线的粗细,待改
频率(Hz) & 1 & 10k & 1M \\
\midrule
输入电阻($\Omega/^\circ$) & 339.719k/-87.84 & 5.6707k/-9.827 & 351.188/-72.377\\
输出电阻($\Omega/^\circ$) & 338.638k/-89.663 & 1.9866k/-1.1228 & 1.9189k/-14.801 \\
\bottomrule
\end{tabular}
\end{table}

\begin{figure}[h]
\centering
\includegraphics[width=12cm]{sample.jpg}
\caption{示例图片}
\label{fig:sample}
\end{figure}
\clearpage





\small
\linespread{1}\selectfont
%正文,五号,中文宋体,英文Time new roman,1倍行距
\chapter*{参考文献}
\noindent
\begin{enumerate}[{[1]}]
\small
	\item 期刊
	\item 网络文档
\end{enumerate}

1.	期刊  作者. 论文名. 刊名, 出版年份, 卷号(期号): 起始-截止页

2.	专著  作者. 书名. 版本(第一版不写). 出版城市: 出版社, 出版年份: 起始-截止页

3.	论文集  论文作者. 论文题目//编者. 论文集名: 其他题名信息. 出版城市(或者会议城市): 出版者, 出版年: 引文起始-截止页码

4.	学位论文  作者. 学位论文题名. 城市: 论文保存单位, 年份

5.	网络文献  作者. 题名[文献类型标志/文献载体标志]. 出版地: 出版者, 出版年(更新日期)[引用日期]. 获取和访问路径

*注意: 作者姓前名后, 超过3名作者列前3名, 后加“, 等”; 英文姓名, 姓前名后, 姓首字母大写, 名缩写; 文献的项目要完整, 各项的顺序和标点要和格式要求一致; 未公开发表的论文、报告不列入正式文献, 如有必要可在正文当页下加注。英文文献格式同上。参考文献在正文中按出现顺序用[1], [2]......在右上角标注, 放在“参考文献”中时, 用[1], [2], ...顺序标注。



\addcontentsline{toc}{chapter}{参考文献}
\fancypagestyle{plain}
{
	\fancyhf{}
	\fancyhead[RE,RO]{参考文献}
	\fancyhead[LE,LO]{北京大学本科生毕业论文}
	\fancyfoot[CO,CE]{~\thepage~}
	\renewcommand{\headrulewidth}{0.7pt}
	\renewcommand{\footrulewidth}{0pt}
}
\fancyhf{}
\fancyhead[RE,RO]{参考文献}
\fancyhead[LE,LO]{北京大学本科生毕业论文}
\fancyfoot[CO,CE]{~\thepage~}
\renewcommand{\headrulewidth}{0.7pt}
\renewcommand{\footrulewidth}{0pt}






\bibliographystyle{plain}
\bibliography{ref}
这是参考“参考文献”,主要用来看引用的顺序,请手动些参考文献或自行写程序,最终编译请删除
\clearpage





\linespread{1}\selectfont
\normalsize
%小四号,中文宋体,英文Time new roman,1倍行距
\chapter*{本科期间的主要工作和成果}

\noindent 本科期间参加的主要科研项目

\noindent 本研基金
\begin{enumerate}
	\item 基金名称. 基金类型. 指导老师. 基金支持年限
\end{enumerate}

\noindent 各种科研项目
\begin{enumerate}
	\item 项目名称. 项目类型
\end{enumerate}

格式下

期刊:

全部作者. 论文名. 期刊名, 出版年份, 卷号(期号): 起始-截止页

会议论文:

全部作者. 论文名. 会议名, 会议举办地, 会议举办时间, 起始-截止页

专利

全部专利申请人. 专利名称. 专利申请号. 专利申请日期. 国别



\addcontentsline{toc}{chapter}{本科期间的主要工作和成果}
\fancypagestyle{plain}
{
	\fancyhf{}
	\fancyhead[RE,RO]{本科期间的主要工作和成果}
	\fancyhead[LE,LO]{北京大学本科生毕业论文}
	\fancyfoot[CO,CE]{~\thepage~}
	\renewcommand{\headrulewidth}{0.7pt}
	\renewcommand{\footrulewidth}{0pt}
}
\fancyhf{}
\fancyhead[RE,RO]{本科期间的主要工作和成果}
\fancyhead[LE,LO]{北京大学本科生毕业论文}
\fancyfoot[CO,CE]{~\thepage~}
\renewcommand{\headrulewidth}{0.7pt}
\renewcommand{\footrulewidth}{0pt}
\clearpage





\linespread{1.5}\selectfont
\normalsize
%正文,小四号,中文宋体,英文Time new roman,1.5倍行距
\chapter*{致谢}



\addcontentsline{toc}{chapter}{致谢}
\fancypagestyle{plain}
{
	\fancyhf{}
	\fancyhead[RE,RO]{致谢}
	\fancyhead[LE,LO]{北京大学本科生毕业论文}
	\fancyfoot[CO,CE]{~\thepage~}
	\renewcommand{\headrulewidth}{0.7pt}
	\renewcommand{\footrulewidth}{0pt}
}
\fancyhf{}
\fancyhead[RE,RO]{致谢}
\fancyhead[LE,LO]{北京大学本科生毕业论文}
\fancyfoot[CO,CE]{~\thepage~}
\renewcommand{\headrulewidth}{0.7pt}
\renewcommand{\footrulewidth}{0pt}





\end{document}